\documentclass{article}

% IMPORT PACKAGES
\usepackage{graphicx}
\graphicspath{{images/}}
\usepackage{blindtext}
\usepackage[T1]{fontenc}
\usepackage[latin9]{inputenc}
\usepackage[a4paper]{geometry}
\geometry{verbose,tmargin=2cm,bmargin=2cm,lmargin=1cm,rmargin=1cm}
\setlength{\parskip}{\smallskipamount}
\setlength{\parindent}{0pt}
\usepackage{array}
\usepackage{mathtools}
\usepackage{dsfont}
\usepackage{amsmath}
\usepackage{amssymb}
\usepackage{lmodern}
\usepackage{eqlist}
\usepackage{babel}
% BLOCKS
\usepackage{beamerarticle}
\usepackage[most]{tcolorbox}
%	COMMON COLORS
\definecolor{_light_green}{rgb}{0.36, 0.84, 0.36}
\definecolor{_light_grey}{rgb}{0.90, 0.90, 0.90}
\definecolor{_white}{rgb}{1.0, 1.0, 1.0}
\definecolor{_blue}{rgb}{0.0, 0.0, 1.0}
\definecolor{_light_blue}{rgb}{0.7, 0.9, 1.0}
%	DEFINE BOXES
\newtcolorbox{_block}[1][]{
    colbacktitle=_light_grey,		% title background
    coltitle=black,					% title color
    titlerule=0pt,
    colback=white!50!_light_grey,	% body background
    boxrule=0.4pt,					% content-frame padding (or frame width)
    colframe=black!20!_light_grey,	% frame color
    left=0mm,						% content and title left padding
    arc=0.5pt,						% border-radius
    title={#1},
}
\newtcolorbox{_example}[1][]{
    colbacktitle=_light_green,
    coltitle=black,
    titlerule=0pt,
    colback=white!60!_light_green,
    boxrule=0pt,
    colframe=white,
    left=0mm,
    arc=0.5pt,
    title={#1},
}
\newtcolorbox{_note}[1][]{
    colbacktitle=_light_blue,
    coltitle=black,
    titlerule=0pt,
    colback=white!60!_light_blue,
    boxrule=0pt,
    colframe=white,
    left=0mm,
    arc=0.5pt,
    title={#1},
}
\newtcolorbox{_block_emph}[1][]{
    enhanced,
    frame hidden,
    borderline west={2.0pt}{0pt}{blue!60!white},
    opacityframe=0.0,
    colback=blue!4!white,
    left=2mm,
    arc=0.5pt,
    title={#1},
}

% \usepackage{subfiles}
% \subfile{}




% DOCUMENT

\begin{document}

% short title:
%   \part*{}
%   \part[]{}

\bold{EXAMPLES, TESTING, DUMP SITE; DELETE LATER}

blocks:

\begin{_block}[$\sigma$-Algebra]

$\mathcal{F}\subset\mathcal{P}(\Omega)$ is a \textbf{$\sigma$-Algebra}
$\iff$

P1: $\Omega\in\mathcal{F}$

P2: ...

\end{_block}

notes:

\begin{_note}[Properties]
\begin{itemize}
\item convenient computation for approximate lower bound of time of event occurrence $t\in\mathbb{R}$
\[
\mathbb{P}[X\geq t]=e^{-\lambda t}
\]
\end{itemize}
\end{_note}

without title:

\begin{_note}[]

$\forall i:A_{i}\in\mathcal{F}\implies\bigcap_{i=1}^{\infty}A_{i}\in\mathcal{F}$

\end{_note}

block\_emph:

\begin{_block_emph}
This is a test.
\end{_block_emph}

\title{CS3237 Introduction to Internet of Things \\ Group 4 Report}
\author{Lai Yu Heem, Pinkl Constantin Maxime, Teo Chuan Kai, Wong Chee Hong, Thomm Leon Felix}
\maketitle

% changing sections enumeration style
% styles: \arabic{} \alph{} \Alph{} \roman{} \Roman{}
\renewcommand\thesection{\arabic{section}}
\renewcommand\thesubsection{\thesection.\alph{subsection}}

\section{Abstract}

\textbf{Tino}

\section{Introduction}

\textbf{Tino}

\section{Solution Approach}

\textbf{Chuan Kai}

\subsection{Architecture overview}

\textbf{Chuan Kai}

\subsection{Buoy Hardware}

\textbf{Leon}

\subsection{Phone applications}

\textbf{Tino}

\subsection{Server-side}

\textbf{Chuan Kai}

\subsection{ML model and application}

\textbf{Yu Heem}

\subsection{Front-End}

\textbf{Chuan Kai}

\section{Implementation Details}

\subsection{Turbidity sensor}

\textbf{Chee Hong}

\subsection{Waterproofing}

\textbf{Chee Hong}

\subsection{Power modes and sampling frequencies}

\textbf{Leon}

\subsection{$\text{I}^2\text{C}$ Communication}

\textbf{Leon}

\subsection{Buoy-Server}

\textbf{Chuan Kai}

\subsection{Data collection}

\textbf{Tino}

\section{Experimental Evaluation}

\subsection{Model accuracy}

\textbf{Chuan Kai}

\subsection{Power consumption estimates}

\textbf{Leon}

\section{Challenges and Outlook}

\subsection{Phone reliance}

\textbf{Tino}

\subsection{Data bottleneck}

\textbf{Yu Heem}

\subsection{Statistical methods}

\textbf{Chee Hong}

\subsection{Other use-cases}

\textbf{???}

\subsection{...}

\end{document}